% !TEX root = ../Main.tex

\chapter{General Conclusions}
\label{chap:conclusion}
Classical simulation has been integral to the study and development of quantum computation from its beginnings~\cite{Feynman1982}. Continued development of classical methods has cast light on the requirements for a quantum advantage~\cite{Jozsa2003}, and even guided the development of quantum hardware by excluding potential systems such as NMR quantum computation~\cite{Braunstein1999}. Now, in the NISQ era, classical simulations are also key to quantum supremacy experiments~\cite{Preskill2012,Aaronson2016}. This project considered classical simulation of quantum circuits based on the stabilizer rank method. Stabilizer rank decompositons are of particular interest as they have a clear connection to the notion of non-stabilizer states as a resource for quantum computing, and an immediate interpretation in terms of hardness of classical simulation.\par
% Introduced new clifford simulation method
In Chapter~\ref{chap:stabilizers}, we introduce novel classical simulators for stabilizer circuits, with additional capabilities beyond commonly used existing methods. Our implementations also have performance that is comparable to or improves on existing publicly available tools. It would also be interesting to compare our method for stabilizer inner products with those of~\cite{Garcia2012}, which are currently closed source.\par
Simulating stabilizer circuits has applications in quantum communication~\cite{Bennett1992,Bennett1993}, and in studying encoding and decoding circuits for stabilizer error correcting codes~\cite{Aaronson2004,Gottesman1997}. Decompositions into stabilizer circuits can also be used to simulate universal quantum computations, and the additional information and routines in our classical data structures makes them advantageous for this purpose~\cite{Bravyi2018}.\par
We discuss these kind of decompositions in Chapter~\ref{chap:stabrank}, where we are able to both extend previous results~\cite{Bravyi2015,Bravyi2016} to different species of magic states, and also present techniques for building stabilizer rank decompositions of arbitrary quantum states.\par
Exact stabilizer rank is a non-convex quantity, and has proven difficult to characterise. We present evidence linking exact stabilizer rank to symmetries of the state, in particular with respect to the Clifford group. We also introduce the notion of stabilizer extent, a convex quantity that acts as an upper bound on approximate stabilizer rank, and show that it can be lower-bounded by the stabilizer fidelity.\par
Finally, in Chapter~\ref{chap:simulator}, we combine these two ingredients and show how they can be used to construct classical simulations of quantum circuits. In particular, we show how to perform strong and weak classical simulations, both in the exact case or approximate to within additive error~\cite{Bravyi2018}. The corresponding spatial and temporal complexity of our simulations scales as
\begin{equation}
O\left(\chi \poly\left(n\right)\right),\label{eq:exact_complexity}
\end{equation} 
n the exact case, or as
\begin{equation}
O\left(\chi_{\epsilon}\poly\left(n,\epsilon\right)\right)\label{eq:approximate_complexity}
\end{equation}in the approximate case.\par
This method is especially appealing as its spatial requirements scale only polynomially with the number of qubits, enabling simulations of quantum circuits on large system sizes and with a bounded number of non-Clifford gates tractable even on a personal computer.\ However, our techniques also have a great propensity to be scaled to HPC systems, and we identify some interesting potential optimizations to the simulation method that could improve performance in this context. Finally, we note that the ability to both strongly and weakly simulate quantum circuits means the stabilizer rank methods also have the potential to act as both verifiers and `heavy output generators' in the context of quantum supremacy experiments~\cite{Aaronson2016}.\par
An important caveat to the discussions in this thesis is that all the methods discussed relate to simulating noiseless quantum circuits on pure states. The only method for incorporating noise into these simulations methods is with stochastic sampling of Pauli or Clifford errors, including measurements and resets. Incorporating these kind of operations cannot increase the simulation complexity, as can be seen from the properties of stabilizer rank discussed in Section~\ref{sec:srank_results}. In fact, it is likely that the stabilizer rank would decrease if we included resets and measurements in the noise model. However, sampling noise in this way does incur an overhead in requiring many repetitions of circuit.\par
This restriction to pure circuits appears in tension with the knowledge that noisy quantum circuits can be efficiently simulated clasically. It would thus be interesting to investigate possible extensions of the stabilizer rank method to mixed states. One possible candidate would be simply to employ stabilizer rank decompositions to each term in a pure state decomposition; using this method, we could avoid an additional negativity overhead by using positive decompositions of the mixed state. However, there is no guarantee on how the overall stabilizer rank would behave for such a decomposition.\par
A natural mixed-state analogue of stabilizer rank would appear to be the Robustness of Magic, which can be defined as~\cite{Howard2017}
\begin{equation}
\mathcal{R}_{\mathcal{M}}\left(\rho\right) = \min_{\braket{\phi}} \norm{\va{c}}_{1}\;:\; \rho = \sum_{i}\va{c}_{i}\braket{\phi_{i}}.
\end{equation}
Continuing research of the robustness of magic has looked at characterising the `non-stabilizerness' of noisy quantum channels, and presented several techniques for simulating mixed state quantum computations with this method~\cite{Seddon2019}. However, it is interesting to note that for pure states, it can be shown that $2 \xi\left(\psi\right) -1 \leq \mathcal{R}_{\mathcal{M}}\left(\braket{\psi}\right)$~\cite{Regula2018}. This may suggest that there are potential savings to be found in extended stabilizer extent to the mixed state case.\par
Finally, we briefly consider the consequences of this thesis in terms of the hardness of simulating quantum computation. As discussed in Section~\ref{sec:srank_discussion}, while we have presented various upper bounds on stabilizer rank, few lower bounds are known. While the work of~\cite{Dalzell2017} presents evidence of an exponential lower-bound for $T$-gates, we have shown that operations with large $T$-synthesis costs can in fact have much smaller stabilizer extent.\par
This leaves open the question of what causes the stabilizer rank of a system to grow exponentially, and in what cases it grows at most polynomially in the number of non-Clifford gates. Indeed, from Equations~\ref{eq:exact_complexity} and~\ref{eq:approximate_complexity}, the stabilizer rank method is explicitly capable of efficiently classically simulating any circuit with this property. However, the results presented in this thesis are often for individual magic states or non-Clifford gates, and extended to circuits through the submultiplicativity of stabilizer rank. Thus, even while we can conceive of states with small stabilizer extent $\sim 1$, their approximate stabilizer rank would still exhibit exponential growth. The existence of families of state of gate that admit such an efficient stabilizer rank simulation is an important open question. The results of~\cite{Qassim2019} represent an important step in improving over these multiplicative bounds, and their application has the potential to greatly extend the range of circuits accessible to the stabilizer rank method. Work in this direction could also potentially help to close the gap between the bounds of~\cite{Dalzell2017}, and the decompositions given in this thesis.