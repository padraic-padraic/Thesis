% !TEX root = ../Main.tex

\chapter{Introduction}\label{chap:introduction}
% Quantum computers are emerging technology
% Widely expected to offer exponential advatnage over classical devices
% By definition, should not be clasically simulable
% Notion of this gap not yet understood
% Classical simulation probes this boundary
Over the past 10 years, quantum computation has rapidly transitioned from a field of research into a burgeoning industry, drawing attention from national governments~\cite{UKNQTP,QuantumFlagship}, and private enterprise alike[citations?]. 
\section{Foundations of Quantum Computing}
% Feynmann, physics of computation link, MIT conference and physical turing machines
% Cite also the quote from tha Gil Kalai paper
% Deutsch, and the Church-Turing-Deutsch hypothesis

\subsection{Complexity Theory and Quantum Advantage}a
% Deutsch Jozsa, hints of speedup
% Shor and grover search
% bernstein vazirani and bounds on NP hard problems
% Connections with problems like #P from ashley, IQP circuits etc.

\section{Classical Descriptions of Quantum Computation}
% statevector methods & p-blockedness
% Density matrix picture
% Tensor network descriptions
% Other methods like the binary search trees JMK thingy

\subsection{Efficient Descriptions of Restricted Models}
% Prop D note from Jozsa et al
% E.g. stabilizer states, explore in detail
% Also contextuality etc
% Also Match gates, very different picture

\subsection{Resource Theories of Quantum Computing}
% Ways of quantifying Prop D

\section{From Classical Descriptions to Simulation}
\subsection{Notions of Classical Simulation}
%Exact and approximate, strong and weak, etc.
% Types of approximation
% Hakop paper about connecting estimating probabilties and sampling

\subsection{Simulation and Quantum Advtantage}
% Quantum supremacy notions
% Not useful task but related to CTD 



