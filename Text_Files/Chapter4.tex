% !TEX root = ../Main.tex

\chapter[Simulating Quantum Circuits with Stabilizer Rank]{Simulating Quantum Circuits with the\\ Stabilizer Rank Method}
\label{chap:simulator}
% Look at combining the previous two chapters
% Build stabilizer decompositions, then simulate circuits efficiently for each term
% Here, present methods for operating on the decompositions, for practical simulation
\section{Introduction}
% Can build simulators based on stabilizer decompositions
% Need routines for maintaining the decomposition and then combining output efficiently
% E.g. Stabilizer Frames method but, as discussed, leads to typcally larger decomposition
% BSS/BG => Technqiues for computing output variables with stabilizer decompositions
% Cf. Other classical simulators e.g. Tensor Networks & Rollright, current SotA
% Mention connection to Quantum Development Tools => Circ, Qiskit, used in prototyping and verification
\section{Results}
% Begin by discussing the simulator
\subsection{Methods for Manipulating Stabilizer Decompositions}
\subsubsection*{Building Decompositions}
\subsubsection*{Output Variables}
% Implement an optimised version of Norm Estimation
% Compute Individual Ampluitudes approximately
% Can be extended to a Monte Carlo Method
\subsubsection*{Implementation and Parallelization}
% Extend the simulators of Chapter 2 w/ wrapper
% Swappable CH/DCH methids
% Building Decompositions and Sampling Outputs is intrinsically trivially parallelizable
% Locally parallised w/ OpenMP
% Distributable w/ MPI
% Requires some synchronisation, especially for sampling
\par
\large{\itshape{Integration with \texttt{Qiskit-Aer}}}
% Also possible to integrate this runner w/ Qiskit Aer
% Circuits parsed by qiskit, build decomposition and then do measurements w/ MC
% Includes support for Noise sampling w/ naive method
% Offers improvements over builtin statevector method
\par
\subsection{Simulations of Quantum Circuits}
% Results of simulations performed with this method
\subsubsection*{Hidden Shift Circuits}
% Used to verify the Simulator
% Test both RandomCode and SoC Methods
% Validate results through comparison to existing MATLAB code
\subsubsection*{QAOA}
% Simulation of an algorithm w/ many Z rotations
% Significant Advantage over old methods w/ synthesis
% Verifiable w/ MC method
% Solvable in minutes w/ Parallelization
\subsubsection*{Random Circuit Models}
% Introduce cicuit model, supremacy test
% Clear signature of 'quantumness' in XE
% Look at dimensionality effects
% Look at CZ Schedule effects
\section{Discussion}
% Significant Advantages over BG16, and other `general purpise' simulators
% Further optimization needed to compete w/ methods like ollright
\subsection{Simulating NISQ Circuits}
% Good use case
% Circuits typically limited in depth and n/ qubits
% Increasing noise w/ depth => trade off in accuracy
% Potential to probe e.g. families of ansatz states in VQE and other variational methods
\subsection{Simulating Random Circuits}
% Current approach uses naive T expansion
% Circuits have lots of structure that could be used
% E.g clifford recompilation, just lots of Pauli blocks, find `contractive' clifford expansions
% ALternatively, look at local rearrangements of T gates
% Reducing T count and increased resources could lead to good results!
\subsection{Incorporating Noise}
% Noise typically makes classical simulation easier
% Above certain noise thresholds mixed state computation is efficiently classically simulable
% Noise has no clear model in this simulator
% Could imagine sampling noise w/ clifford gates and measurement => Will in general decrease rank but hard to estiamte actual effect on runtime
% Consider 'Mixed State Extension' => Better asymptotic behaviour than robustness
% Unclear how to construct in practice...
