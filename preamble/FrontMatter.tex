% !TEX root = ../Main.tex

%% YOUR DETAILS

\title{Exploring Quantum Computation Through the Lens of Classical Simulation}
\author{Padraic \textsc{Calpin}}
\department{Department of Physics \& Astronomy}
\supervisor{Prof. Dan \textsc{Browne}}
\date{\today}

\maketitle
\makedeclaration

\begin{abstract} % 300 word limit
It is widely believed that quantum computation has the potential to offer an exponential speedup over classical devices. However, there is currently no definitive proof of this separation in computational power.\par
This separation would in turn imply that quantum circuits cannot be efficiently simulated classically. However, it is well known that certain classes of quantum computations nonetheless admit an efficient classical description. Recent work has also argued that accurate classical simulation of quantum circuits would imply the collapse of the Polynomial Hierarchy, something which is commonly invoked in classical complexity theory as a no-go theorem. This suggests a route for studying this ‘quantum advantage’ through classical simulations.\par
This project looks at the problem of classically simulating quantum circuits through decompositions into stabilizer circuits. These are a restricted class of quantum computation which can be efficiently simulated classically. In this picture, the rank of the decomposition determines the temporal and spatial complexity of simulating the computation.\par
We approach the problem by considering classical simulations of stabilizer circuits, introducing two new representations with novel features compared to previous methods. We then examine techniques for building these so-called ‘stabilizer rank’ decompositions, both exact and approximate.  Finally, we combine these two ingredients to introduce an improved method for classically simulating broad classes of circuits using the stabilizer rank method.
\end{abstract}

\begin{impactstatement}
	UCL theses now have to include an impact statement. \textit{(I think for REF reasons?)} The following text is the description from the guide linked from the formatting and submission website of what that involves. (Link to the guide: {\scriptsize \url{http://www.grad.ucl.ac.uk/essinfo/docs/Impact-Statement-Guidance-Notes-for-Research-Students-and-Supervisors.pdf}})

\begin{quote}
The statement should describe, in no more than 500 words, how the expertise, knowledge, analysis,
discovery or insight presented in your thesis could be put to a beneficial use. Consider benefits both
inside and outside academia and the ways in which these benefits could be brought about.

The benefits inside academia could be to the discipline and future scholarship, research methods or
methodology, the curriculum; they might be within your research area and potentially within other
research areas.

The benefits outside academia could occur to commercial activity, social enterprise, professional
practice, clinical use, public health, public policy design, public service delivery, laws, public
discourse, culture, the quality of the environment or quality of life.

The impact could occur locally, regionally, nationally or internationally, to individuals, communities or
organisations and could be immediate or occur incrementally, in the context of a broader field of
research, over many years, decades or longer.

Impact could be brought about through disseminating outputs (either in scholarly journals or
elsewhere such as specialist or mainstream media), education, public engagement, translational
research, commercial and social enterprise activity, engaging with public policy makers and public
service delivery practitioners, influencing ministers, collaborating with academics and non-academics
etc.

Further information including a searchable list of hundreds of examples of UCL impact outside of
academia please see \url{https://www.ucl.ac.uk/impact/}. For thousands more examples, please see
\url{http://results.ref.ac.uk/Results/SelectUoa}.
\end{quote}
\end{impactstatement}

\begin{acknowledgements}
Acknowledge all the things!
\end{acknowledgements}

\setcounter{tocdepth}{2} 
% Setting this higher means you get contents entries for
%  more minor section headers.

\tableofcontents
% \listoffigures
% \listoftables
% % !TEX root = ../Main.tex

\renewcommand{\nomname}{List of Symbols}
\renewcommand{\nompreamble}{The next list describes several symbols that will be later used within the body of the document.
If not using \texttt{latexmk}, will need to run code \texttt{makeindex Main.nlo -s nomencl.ist -o Main.nls}.
In any case, compile twice 
}

\nomenclature{$c$}{Speed of light in a vacuum inertial frame}
\nomenclature{$\hbar$}{Reduced Planck constant}
% \printnomenclature


