% !TEX root = ../Main.tex

%% YOUR DETAILS

\title{Exploring Quantum Computation Through the Lens of Classical Simulation}
\author{Padraic \textsc{Calpin}}
\department{Department of Physics \& Astronomy}
\supervisor{Prof. Dan \textsc{Browne}}
\date{\today}

\maketitle
\makedeclaration

\begin{abstract} % 300 word limit
It is widely believed that quantum computation has the potential to offer an exponential speedup over classical devices. However, there is currently no definitive proof of this separation in computational power.\par
This separation would in turn imply that quantum circuits cannot be efficiently simulated classically. However, it is well known that certain classes of quantum computations nonetheless admit an efficient classical description. Recent work has also argued that accurate classical simulation of quantum circuits would imply the collapse of the Polynomial Hierarchy, something which is commonly invoked in classical complexity theory as a no-go theorem. This suggests a route for studying this ‘quantum advantage’ through classical simulations.\par
This project looks at the problem of classically simulating quantum circuits through decompositions into stabilizer circuits. These are a restricted class of quantum computation which can be efficiently simulated classically. In this picture, the rank of the decomposition determines the temporal and spatial complexity of simulating the computation.\par
We approach the problem by considering classical simulations of stabilizer circuits, introducing two new representations with novel features compared to previous methods. We then examine techniques for building these so-called ‘stabilizer rank’ decompositions, both exact and approximate.  Finally, we combine these two ingredients to introduce an improved method for classically simulating broad classes of circuits using the stabilizer rank method.
\end{abstract}

\begin{impactstatement}
This thesis is focused on classical simulation of quantum computing, an important tool both for understanding the theoretical separation between quantum and classical computations, and for developing quantum software in an era where access to actual hardware is still significantly limited.\par
The impact of this thesis is to significantly extend the notion of stabilizer rank, a classical description of quantum systems that has received considerable interest in the community. As part of this, we introduce a novel method for simulating quantum circuits. This method enables simulations of near-term quantum computations on much larger system sizes than previous publicly available tools. For example, we demonstrate simulations of the Quantum Approximate Optimization Algorithm on $50$ qubits, using a personal computer.\par
We have developed open-source software implementations of our work, enabling researchers in academia and industry to apply our methods to simulate quantum computations. We have also integrated the work with IBM's Qiskit framework~\cite{Qiskit}, making the benefits of our method immediately available to the large international community of quantum software developers already experimenting with the platform, including enthusiasts and educators.\par
Our work has been widely disseminated through the arXiv, has been cited $16$ times, and was the third highest-rated paper for 2018 on Sci-Rate, an open-access community for discussion and recommendation of preprints\footnote{\url{https://scirate.com/?date=2019-01-01&range=365}}. It has since been published in Quantum, a high-quality open access journal run by members of the quantum information community, and subsequent papers have been published building on our results~\cite{Qassim2019,Yifei2019}.\par
As well as simulations of universal quantum circuits, we developed two novel methods for simulating a common class of quantum circuit called stabilizer circuits. Our software implementations show generally better performance than current popular methods, and are available to the community through open-source.\footnote{\url{https://github.com/padraic-padraic/StabilizerSim}}\par
Our results on the stabilizer rank also impact attempts in the community to understand non-stabilizer states as a `resource' for quantum advantage over classical computations. The exact origins of quantum speedup are still not entirely understood, and our results on both exact and approximate stabilizer rank decompositions have consequences for the design of quantum algorithms and quantum computing architectures. For example, the quantity of stabilizer extent introduced in Chapter 3 has since been applied to the problem of synthesising quantum gates from a limited gate-set~\cite{Beverland2019}.
\end{impactstatement}

\begin{acknowledgements}
To do...
\end{acknowledgements}

\setcounter{tocdepth}{2} 
% Setting this higher means you get contents entries for
%  more minor section headers.

\tableofcontents
% \listoffigures
% \listoftables
% !TEX root = ../Main.tex

\renewcommand{\nomname}{List of Symbols}
\renewcommand{\nompreamble}{The next list describes several symbols that will be later used within the body of the document.
If not using \texttt{latexmk}, will need to run code \texttt{makeindex Main.nlo -s nomencl.ist -o Main.nls}.
In any case, compile twice 
}

\nomenclature{$c$}{Speed of light in a vacuum inertial frame}
\nomenclature{$\hbar$}{Reduced Planck constant}
\printnomenclature


