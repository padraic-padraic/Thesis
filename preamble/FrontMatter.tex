% !TEX root = ../Main.tex

%% YOUR DETAILS

\title{Exploring Quantum Computation Through the Lens of Classical Simulation}
\author{Padraic \textsc{Calpin}}
\department{Department of Physics \& Astronomy}
\supervisor{Prof.~Dan \textsc{Browne}}
\date{\today}

\maketitle
\makedeclaration

\begin{abstract} % 300 word limit
It is widely believed that quantum computation has the potential to offer an exponential speedup over classical devices. However, there is currently no definitive proof of this separation in computational power.\par
Such a separation would in turn imply that quantum circuits cannot be efficiently simulated classically. However, it is well known that certain classes of quantum computations nonetheless admit an efficient classical description. Recent work has also argued that efficient classical simulation of quantum circuits would imply the collapse of the Polynomial Hierarchy, something which is commonly invoked in classical complexity theory as a no-go theorem. This suggests a route for studying this ‘quantum advantage’ through classical simulations.\par
This project looks at the problem of classically simulating quantum circuits through decompositions into stabilizer circuits. These are a restricted class of quantum computation which can be efficiently simulated classically. In this picture, the rank of the decomposition determines the temporal and spatial complexity of the simulation.\par
We approach the problem by considering classical simulations of stabilizer circuits, introducing two new representations with novel features compared to previous methods. We then examine techniques for building these so-called ‘stabilizer rank’ decompositions, both exact and approximate.  Finally, we combine these two ingredients to introduce an improved method for classically simulating broad classes of circuits using the stabilizer rank method.
\end{abstract}

\begin{impactstatement}
This thesis is focused on classical simulation of quantum computing, an important tool both for understanding the theoretical separation between quantum and classical computations, and for developing quantum software in an era where access to actual hardware is still significantly limited.\par
The impact of this thesis is to significantly extend the notion of stabilizer rank, a classical description of quantum systems that has received considerable interest in the community. As part of this, we introduce a novel method for simulating quantum circuits. This method enables simulations of near-term quantum computations on much larger system sizes than previous publicly available tools. For example, we demonstrate simulations of the Quantum Approximate Optimization Algorithm on $50$ qubits, using a personal computer.\par
We have developed open-source software implementations of our work, enabling researchers in academia and industry to apply our methods to simulate quantum computations. We have also integrated the work with IBM's Qiskit framework~\cite{Qiskit}, making the benefits of our method immediately available to the large international community of quantum software developers already experimenting with the platform, including enthusiasts and educators.\par
% Our work has been widely disseminated through the arXiv, has been cited $16$ times, and was the third highest-rated paper for 2018 on Sci-Rate, an open-access community for discussion and recommendation of preprints\footnote{\url{https://scirate.com/?date=2019-01-01&range=365}}. It has since been published in Quantum, a high-quality open access journal run by members of the quantum information community, and subsequent papers have been published building on our results~\cite{Qassim2019,Yifei2019}.\par
As well as simulations of universal quantum circuits, we developed two novel methods for simulating a common class of quantum circuit called stabilizer circuits. Our software implementations show generally better performance than current popular methods, and are available to the community through open-source.\footnote{\url{https://github.com/padraic-padraic/StabilizerSim}}\par
Our results on the stabilizer rank also impact attempts in the community to understand non-stabilizer states as a `resource' for quantum advantage over classical computations. The exact origins of quantum speedup are still not entirely understood, and our results on both exact and approximate stabilizer rank decompositions have consequences for the design of quantum algorithms and quantum computing architectures. For example, the quantity of stabilizer extent introduced in Chapter 3 has since been applied to the problem of synthesising quantum gates from a limited gate-set~\cite{Beverland2019}.
\end{impactstatement}

\begin{acknowledgements}
There a many people I wish to thank over the course of my four years at UCL.\par
First and foremost, I would like to thank my supervisor, Professor Dan Browne, for all his support and guidance, and whose help and encouragement was a big part of growing a Master's project into an international collaboration. I also want to thank my second supervisor Alessio Serafini, for his support in my PhD upgrade and for giving me the opportunity to participate in teaching and marking.\par
I am very grateful to our collaborators in this project; Earl Campbell, for his knowledge and helpful discussions; Mark Howard and David Gosset, for their help in developing simulations for the paper; and Sergey Bravyi, for his insight and for hosting me during my visit to IBM Yorktown.\par
I would also like to thank my colleagues and friends at UCL, especially the Browne group, and Cohort 2 of the Delivering Quantum Technologies CDT, who have been there at conferences, and after-lunch coffee breaks and much needed after-work drinks.\par
I have benefited greatly from the support and opportunities given to me by the DQT CDT, and would especially like to thank Lopa Murgai, without whom the experience wouldn't have been the same.\par
I would also like to acknowledge the UCL Research Information Technology Services department, for their training and advice that have been key to my PhD, and to acknowledge the use of the UCL Myriad (Myriad@UCL) and Legion (Legion@UCL) High Throughput Computing Facilities, and associated support services, in the completion of this work.\par
Finally, I would like to thank my friends, especially Yasmin and Kier, for many long and helpful chats; my family, Fintan, Stella and Dermott, who have always been there for me, encouraged and supported me, especially over big dinners and late nights; and Polona --- you have been with me throughout everything these four years, and without your patient ear (even when I’m poorly explaining things), sage advice and kind words I wouldn't have been able to do this.\par
I would like to dedicate this thesis to my grandmother Mary, and to the memory of my grandparents Bob, Jack and Lil.
\end{acknowledgements}

\setcounter{tocdepth}{2} 
% Setting this higher means you get contents entries for
%  more minor section headers.

\tableofcontents
% \listoffigures
% \listoftables
% !TEX root = ../Main.tex

\renewcommand{\nomname}{List of Symbols}
\renewcommand{\nompreamble}{The following is a list of some common symbols and notations used in this thesis.
\par
We also make us of common abbreviations for computational complexity classes, which we denote in bold-faced type e.g. $\P$
}

\nomenclature[01]{$n$}{Number of qubits in a given quantum circuit or computation.}
\nomenclature[02]{$\omega$}{The $8$th root of unity, $\omega=\frac{1+\mathi}{\sqrt{2}}$.}
\nomenclature[03]{$\mathbb{Z}_{n}$}{The space of all positive integers modulo $n$.}
\nomenclature[04]{$\va{x}$}{A vector quantity, typically a binary vector.}
\nomenclature[05]{$\va{e}_{i}$}{A binary vector that is all-zero except for the $i$th entry.}
\nomenclature[06]{$\ket{\va{x}}$}{A computational basis state defined by the binary vector $\va{x}$.}
\nomenclature[07]{$\ket{\psi}$}{An arbitrary quantum state.}
\nomenclature[08]{$\ket{\phi},\ket{\varphi}$}{A stabilizer state.}
\nomenclature[09]{$U$}{An arbitrary unitary operator.}
\nomenclature[10]{$V$}{An arbitrary operator belonging to the Clifford group.}
\nomenclature[11]{$\mathcal{P}_{n}$}{The $n$-qubit Pauli group}
\nomenclature[12]{$\mathcal{C}_{j}$}{The set of unitaries belonging to level $j$ of the Clifford hierarchy. Also a group for $j\leq 3$.}
\nomenclature[13]{$p_{U}\left(\va{x}\right)$}{The probability associated with the basis state $\va{x}$ in the state $U\ket{\va{0}}$.}
\nomenclature[14]{$\chi$}{The stabilizer rank, the rank of a stabilizer state decomposition. Typically, this refers to an exact decomposition.}
\nomenclature[15]{$\chi_{\epsilon}$}{The rank of an aproxiamate stabilizer state decomposition with maximum imprecision $\epsilon$.}
\nomenclature[16]{$\xi\left(\psi\right)$}{The stabilizer extent of a state $\ket{\psi}$.}
\nomenclature[17]{$O(g\left(n\right))$}{An asymptotic scaling bounded above by $k g\left(n\right)$ for some constant $k>0$}\nomenclature[18]{$\Theta(g\left(n\right))$}{An asymptotic scaling bounded above by $k_{1} g\left(n\right)$ and below by $k_{2}g\left(n\right)$ for some constants $k_{1,2}>0$}
\nomenclature[19]{$\Omega(g\left(n\right))$}{An asymptotic scaling that grows larger than $k g\left(n\right)$ for some constant $k>0$ and some value of $n$.}
\printnomenclature


